\section{Vorgehen}

In diesem Kapitel wird das Vorgehen bei der Entwicklung erläutert.

\subsection{Unittests}

Bei der Entwicklung wurden von Beginn an Unittests verwendet um die Funktionsweise des HtmlFilters zu überprüfen.
Es wurden für alle Tests unter \url{https://www.owasp.org/index.php/XSS_Filter_Evasion_Cheat_Sheet} ein entsprechender JUnitTest erstellt.
Dadurch konnte bei einer Anpassung eines Filters schnell überprüft werden, ob der bösartige HTML Code noch entfernt wird.

\subsection{FindBugs}

Während der gesamten Entwicklung wurde FindBugs verwendet, um mögliche Fehler bereits 
früh identifizieren zu können. Am Ende der Entwicklung meldete Findbugs keinerlei Fehler mehr.

\subsubsection{JSR-305}

Um FindBugs die Arbeit zu erleichtern, haben wir im gesamten Sourcecode die Annotationen aus dem JSR-305 
verwendet.
\subsection{Robust Programming}

Bei der Entwicklung wurde Wert darauf gelegt, Robuste Java Klassen zu entwickeln. Aus diesem Grund 
sind alle Klassen, welche nicht zur Ableitung gedacht sind \textit{final}. Bei den Abstrakten Klassen,
welche zur Ableitung designt sind, sind alle Methoden \textit{final}, die nicht von Subklassen überschrieben
werden sollen. Bei allen Klassen wurde die Sichtbarkeit der Felder minimiert, sowie auf innere Klassen 
verzichtet, um die Sichbarkeitseinschränkung nicht zu umgehen. Wenn immer möglich, wurden die 
Konstruktoren \textit{private} deklariert, und Factory-Methoden implementiert. Dies ist insbesondere
bei den Filter implementierungen zu sehen, welche meistens eine \textit{createDefault} Methode 
bereitstellen.\newline

